\documentclass{article}

\usepackage[utf8]{inputenc}
\usepackage[francais]{babel}
\usepackage{amsmath}
\usepackage{tikz}

\usetikzlibrary{calc}

\title{Algèbre de chemins}
\author{Merwan Achibet}
\date{}

\begin{document}

\newcommand{\mmin}{\ensuremath{\operatorname{min}}}
\newcommand{\mmax}{\ensuremath{\operatorname{max}}}
\newcommand{\mMIN}{\ensuremath{\operatorname{MAX}}}
\newcommand{\mMAX}{\ensuremath{\operatorname{MAX}}}

\maketitle

\section{Montrer que $(R^+, \mmax, \mmin, 0,
  \infty)$ est un semi-anneau idempotent}

On veut démontrer que $(R^+, \mmax, \mmin, 0, \infty)$ est un
semi-anneau idempotent. Pour ce faire, on doit prouver que :

\begin{enumerate}
  \item{$(R^+, \mmax, 0)$ est un monoïde commutatif}
  \item{$(R^+, \mmin, \infty)$ est un monoïde}
  \item{l'opération $\mmin$ est distributive par rapport à $\mmax$}
  \item{l'élément $0$ est absorbant pour l'opération $\mmin$}
\end{enumerate}

Pour les démonstrations qui suivent, on prend $a, b, c \in R^+$.

\subsection{$(R^+, \mmax, 0)$ est un monoïde commutatif}

$$ \mmax(a, b) = \mmax(b, a)$$
\begin{center}et\end{center}
$$ \mmax(a, 0) = \mmax(0, a) = a $$

$(R^+, \mmax, 0)$ est donc un monoïde commutatif.

\subsection{$(R^+, \mmin, \infty)$ est un monoïde}

$$ \mmin(a, \infty) = \mmin(\infty, a) = a$$

$(R^+, \mmin, \infty)$ est donc un monoïde.

\subsection{l'opération $\mmin$ est distributive par rapport à $\mmax$}

Pour que cette opération soit distributive, il faut que

$$
\mmin(a, \mmax(b,c))
=
\mmax(\mmin(a,b),\mmin(a,c))
$$

et que

$$
\mmin(\mmax(a,b),c)
=
\mmax(\mmin(a,c),\mmin(a,b))
$$

On étudie les six cas de figure possibles.

\subsubsection{$a \geq b \geq c$}
\subsubsection{$a \geq c \geq b$}
\subsubsection{$b \geq a \geq c$}
\subsubsection{$b \geq c \geq a$}
\subsubsection{$c \geq a \geq b$}
\subsubsection{$c \geq b \geq a$}

\subsection{l'élément  $0$ est absorbant pour l'opération $\mmin$}

$$
\mmin(a,0) = \mmin(0,a) = 0
$$

L'élément $0$ est donc absorbant pour $\mmin$.

Les quatres conditions précedemment citées sont validées, $(R^+,
\mmax, \mmin, 0, \infty)$ est donc un
semi-anneau idempotent.

La matrice d'adjacence est la suivante. On considère que pour tout $i
\in \{1,\dots,8\}$, $A_{ii} = \infty$ puisque la capacité

$$
A = \bordermatrix{
    & 1 & 2 & 3 & 4 & 5 & 6 & 7 & 8 \cr
  1 & \infty & 12 & 14 & 10 & 0 & 0 & 0 & 0 \cr
  2 & 12 & \infty & 0 & 17 & 8 & 0 & 0 & 0 \cr
  3 & 14 & 0 & \infty & 5 & 0 & 3 & 0 & 0 \cr
  4 & 10 & 17 & 5 & \infty & 11 & 6 & 15 & 0 \cr
  5 & 0 & 8 & 0 & 11 & \infty & 0 & 18 & 11 \cr
  6 & 0 & 0 & 3 & 6 & 0 & \infty & 4 & 15 \cr
  7 & 0 & 0 & 0 & 15 & 18 & 4 & \infty & 9 \cr
  8 & 0 & 0 & 0 & 0 & 11 & 15 & 9 & \infty
}
$$

\section{Appliquer l'algorithme de Warshall pour calculer $A^*$}

$$
A^* = \bordermatrix{
    & 1 & 2 & 3 & 4 & 5 & 6 & 7 & 8 \cr
  1 & \infty & 12 & 14 & 12 & 12 & 11 & 12 & 11 \cr
  2 & 12 & \infty & 12 & 17 & 15 & 11 & 15 & 11 \cr
  3 & 14 & 12 & \infty & 12 & 12 & 11 & 12 & 11 \cr
  4 & 12 & 17 & 12 & \infty & 15 & 11 & 15 & 11 \cr
  5 & 12 & 15 & 12 & 15 & \infty & 11 & 18 & 11 \cr
  6 & 11 & 11 & 11 & 11 & 11 & \infty & 11 & 15 \cr
  7 & 12 & 15 & 12 & 15 & 18 & 11 & \infty & 11 \cr
  8 & 11 & 11 & 11 & 11 & 11 & 15 & 11 & \infty
}
$$

\section{Graphe de $A^*$}

\begin{tikzpicture}

  \centering

  \tikzstyle{arete}=[draw,circle]
  \tikzstyle{capa}=[draw,rectangle,fill=white]

  \foreach \i in {1,...,8} {
    \node[circle, draw] at ({\i*(360/8)+45}:5cm) (\i) {\i};
    \coordinate (c\i) at ({\i*(360/8)+45}:5cm);
  }

  \newcommand{\link}[4]{
    \draw (#1) -- (#2);
    \node[draw,#4] at ($(1,1)+(3,3)$) {#3}; % TODO: partway
  }

  \link{1}{2}{12}{blue}
  \link{1}{3}{14}{red}
  \link{1}{4}{12}{blue}
  \link{1}{5}{12}{blue}
  \link{1}{6}{11}{blue}
  \link{1}{7}{12}{blue}
  \link{1}{8}{11}{blue}

  \link{2}{3}{12}{blue}
  \link{2}{4}{17}{blue}
  \link{2}{5}{15}{blue}
  \link{2}{6}{11}{blue}
  \link{2}{7}{15}{blue}
  \link{2}{8}{11}{blue}

  \link{3}{4}{12}{blue}
  \link{3}{5}{12}{blue}
  \link{3}{6}{11}{blue}
  \link{3}{7}{12}{blue}
  \link{3}{8}{11}{blue}

  \link{4}{5}{15}{blue}
  \link{4}{6}{11}{blue}
  \link{4}{7}{15}{blue}
  \link{4}{8}{11}{blue}

  \link{5}{6}{11}{blue}
  \link{5}{6}{18}{blue}
  \link{5}{6}{11}{blue}

  \link{6}{7}{11}{blue}
  \link{6}{8}{15}{blue}

  \link{7}{8}{11}{blue}

\end{tikzpicture}

\section{Appliquer l'algorithme de Dijkstra }

On applique manuellement l'algorithme de Dijkstra :

Initialisation

$ \pi(1) = \infty $\\
$ \pi(2) = \dots = \pi(8) = 0$\\
$ T = \{1,2,3,4,5,6,7,8\}$

Execution

1/
$\forall j \in T$, $i = \mMAX(\pi_j) = 1$\\
$ T = \{2,3,4,5,6,7,8\}$

$\pi(2) = \mmax(\pi(2), \mmin(\pi(1), A_{12})) = \mmax(0, \mmin(\infty,12)) = 12$\\
$\pi(3) = \mmax(\pi(3), \mmin(\pi(1), A_{13})) = \mmax(0, \mmin(\infty,14)) = 14$\\
$\pi(4) = \mmax(\pi(4), \mmin(\pi(1), A_{14})) = \mmax(0, \mmin(\infty,10)) = 10$\\
$\pi(5) = \mmax(\pi(5), \mmin(\pi(1), A_{15})) = \mmax(0, \mmin(\infty,0)) = 0$\\
$\pi(6) = \mmax(\pi(6), \mmin(\pi(1), A_{16})) = \mmax(0, \mmin(\infty,0)) = 0$\\
$\pi(7) = \mmax(\pi(7), \mmin(\pi(1), A_{17})) = \mmax(0, \mmin(\infty,0)) = 0$\\
$\pi(8) = \mmax(\pi(8), \mmin(\pi(1), A_{18})) = \mmax(0, \mmin(\infty,0)) = 0$

2/
$\forall j \in T$, $i = \mMAX(\pi_j) = 3$\\
$ T = \{2,4,5,6,7,8\}$

$\pi(2) = \mmax(\pi(2), \mmin(\pi(3), A_{32})) = \mmax(12, \mmin(14,0)) = 12$\\
$\pi(4) = \mmax(\pi(4), \mmin(\pi(3), A_{34})) = \mmax(10, \mmin(14,5)) = 10$\\
$\pi(5) = \mmax(\pi(5), \mmin(\pi(3), A_{35})) = \mmax(0, \mmin(14,0)) = 0$\\
$\pi(6) = \mmax(\pi(6), \mmin(\pi(3), A_{36})) = \mmax(0, \mmin(14,3)) = 3$\\
$\pi(7) = \mmax(\pi(7), \mmin(\pi(3), A_{37})) = \mmax(0, \mmin(14,0)) = 0$\\
$\pi(8) = \mmax(\pi(8), \mmin(\pi(3), A_{38})) = \mmax(0, \mmin(14,0)) = 0$

3/
$\forall j \in T$, $i = \mMAX(\pi_j) = 2$\\
$ T = \{4,5,6,7,8\}$

$\pi(4) = \mmax(\pi(4), \mmin(\pi(2), A_{24})) = \mmax(10, \mmin(12,17)) = 12$\\
$\pi(5) = \mmax(\pi(5), \mmin(\pi(2), A_{25})) = \mmax(0, \mmin(12,8)) = 8$\\
$\pi(6) = \mmax(\pi(6), \mmin(\pi(2), A_{26})) = \mmax(3, \mmin(12,0)) = 3$\\
$\pi(7) = \mmax(\pi(7), \mmin(\pi(2), A_{27})) = \mmax(0, \mmin(12,0)) = 0$\\
$\pi(8) = \mmax(\pi(8), \mmin(\pi(2), A_{28})) = \mmax(0, \mmin(12,0)) = 0$

4/
$\forall j \in T$, $i = \mMAX(\pi_j) = 4$\\
$ T = \{5,6,7,8\}$

$\pi(5) = \mmax(\pi(5), \mmin(\pi(4), A_{45})) = \mmax(8, \mmin(12,11)) = 11$\\
$\pi(6) = \mmax(\pi(6), \mmin(\pi(4), A_{46})) = \mmax(3, \mmin(12,6)) = 6$\\
$\pi(7) = \mmax(\pi(7), \mmin(\pi(4), A_{47})) = \mmax(0, \mmin(12,15)) = 12$\\
$\pi(8) = \mmax(\pi(8), \mmin(\pi(4), A_{48})) = \mmax(0, \mmin(12,0)) = 0$

5/
$\forall j \in T$, $i = \mMAX(\pi_j) = 7$\\
$ T = \{5,6,8\}$

$\pi(5) = \mmax(\pi(5), \mmin(\pi(7), A_{75})) = \mmax(11, \mmin(12,18)) = 12$\\
$\pi(6) = \mmax(\pi(6), \mmin(\pi(7), A_{76})) = \mmax(6, \mmin(12,4)) = 6$\\
$\pi(8) = \mmax(\pi(8), \mmin(\pi(7), A_{78})) = \mmax(0, \mmin(12,9)) = 9$

6/
$\forall j \in T$, $i = \mMAX(\pi_j) = 5$\\
$ T = \{6,8\}$

$\pi(6) = \mmax(\pi(6), \mmin(\pi(5), A_{56})) = \mmax(6, \mmin(12,0)) = 6$\\
$\pi(8) = \mmax(\pi(8), \mmin(\pi(5), A_{58})) = \mmax(9, \mmin(12,11)) = 11$

7/
$\forall j \in T$, $i = \mMAX(\pi_j) = 8$\\
$ T = \{6\}$

$\pi(6) = \mmax(\pi(6), \mmin(\pi(8), A_{86})) = \mmax(6, \mmin(11,15)) = 11$\\

$T = \emptyset$ alors on a terminé.

BLABLA

\section{Jacobi}

\section{Jacobi amélioré}

\end{document}
