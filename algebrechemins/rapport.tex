\documentclass{article}

\usepackage[utf8]{inputenc}
\usepackage[francais]{babel}
\usepackage{amsmath}
\usepackage{tikz}

\title{Algèbre de chemins}
\author{Merwan Achibet}
\date{}

\begin{document}

\maketitle

\section{Semi-anneau}

On veut démontrer que $(R^+, \operatorname{max}, \operatorname{min}, 0, \infty)$ est
un semi-anneau idempotent. Pour ce faire, on doit prouver que :

\begin{enumerate}
  \item{$(R^+, \operatorname{max}, 0)$ est un monoïde commutatif}
  \item{$(R^+, \operatorname{min}, \infty)$ est un monoïde}
  \item{l'opération $\operatorname{min}$ est distributive par rapport à $\operatorname{max}$}
  \item{l'élément $0$ est absorbant pour l'opération $\operatorname{min}$}
\end{enumerate}

Pour les démonstrations qui suivent, on prend $a, b, c \in R^+$.

\subsection{$(R^+, \operatorname{max}, 0)$ est un monoïde commutatif}

$$
\operatorname{max}(a, b) = \operatorname{max}(b, a)
$$

et

$$
\operatorname{max}(a, 0) = \operatorname{max}(0, a) = a
$$

donc il s'agit bien d'un monoïde commutatif.

\subsection{$(R^+, \operatorname{min}, \infty)$ est un monoïde}

$$
\operatorname{min}(a, \infty) = \operatorname{min}(\infty, a) = a
$$

donc il s'agit bien d'un monoïde

\subsection{l'opération $\operatorname{min}$ est distributive par rapport à $\operatorname{max}$}

Pour que cette opération soit distributive, il faut que

$$
\operatorname{min}(a, \operatorname{max}(b,c))
=
\operatorname{max}(\operatorname{min}(a,b),\operatorname{min}(a,c))
$$

et que

$$
\operatorname{min}(\operatorname{max}(a,b),c)
=
\operatorname{max}(\operatorname{min}(a,c),\operatorname{min}(a,b))
$$

On étudie les six cas de figure possibles.

\subsubsection{$a \geq b \geq c$}
\subsubsection{$a \geq c \geq b$}
\subsubsection{$b \geq a \geq c$}
\subsubsection{$b \geq c \geq a$}
\subsubsection{$c \geq a \geq b$}
\subsubsection{$c \geq b \geq a$}

\subsection{l'élément  $0$ est absorbant pour l'opération $\operatorname{min}$}

$$
\operatorname{min}(a,0) = \operatorname{min}(0,a) = 0
$$

L'élément $0$ est donc absorbant pour $\operatorname{min}$.

Les quatres conditions précedemment citées sont validées, $(R^+,
\operatorname{max}, \operatorname{min}, 0, \infty)$ est donc un
semi-anneau idempotent.

\subsection{Matrice d'adjacence}

$$
T = \bordermatrix{
    & 1 & 2 & 3 & 4 & 5 & 6 & 7 & 8 \cr
  1 & 0 & 12 & 14 & 10 & \infty & \infty & \infty & \infty \cr
  2 & 12 & 0 & \infty & 17 & 8 & \infty & \infty & \infty \cr
  3 & 14 & \infty & 0 & 5 & \infty & 3 & \infty & \infty \cr
  4 & 10 & 17 & 5 & 0 & 11 & 6 & 15 & \infty \cr
  5 & \infty & 8 & \infty & 11 & 0 & \infty & 18 & 11 \cr
  6 & \infty & \infty & 3 & 6 & \infty & 0 & 4 & 15 \cr
  7 & \infty & \infty & \infty & 15 & 18 & 4 & 0 & 9 \cr
  8 & \infty & \infty & \infty & \infty & 11 & 15 & 9 & 0
}
$$

\section{Warshall}

\section{A*}

\section{Dijkstra}

\section{Jacobi}

\section{Jacobi amélioré}

\end{document}
