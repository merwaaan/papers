\documentclass[12pt]{article}

\usepackage[utf8]{inputenc}
\usepackage{url}
\usepackage{xcolor}

\definecolor{bg}{RGB}{85,98,112}

\renewcommand{\baselinestretch}{1.5}

\title{Are Large Software Companies Bound to Adopt the Free Software Model?}
\author{Merwan Achibet $\;$ \textendash $\;$  Université du Havre}
\date{}

\begin{document}

\maketitle

\section*{Abstract}

With an ever-increasing computer accessibility and a huge available
market, large software companies typically release proprietary
softwares, but with the growth of the free and open source movement,
we have to ask ourselves if this situation is bound to change. An
overview of the advantages of free software and a presentation of
several business models compatible with free software shows that it
has the ability to attract software companies. Even if free software
and proprietary software are meant to cohabit without one model
overthrowing the other, we observe a clear increase in the number of
companies involved with free projects and we explain that the main
brake to this growth comes from the intrinsic openness of free
software which does not allow companies to hide their innovation from
competitors.

\section*{Keywords}

Free software, open source, proprietary software, software companies

\newpage

\tableofcontents

\newpage

\section{Context}

\subsection{Definition}

Before pondering about the interests that large software companies
might express for the free software model, we have to clearly define
it. The concept has a large spectrum of significance and definitions
are many, often separated by subtle differences. Nevertheless, they
all share the same basis \cite{sers}. Firstly, the intrinsic quality
of this kind of software comes from its openness: its source code is
available for all, and above all, editable by all. There are thus no
restriction on the amount of hacking that such a software can
endure. Secondly, the available source code is candidate to
redistribution, as long as the new software perpetuate the original
licence. A differencation must be made between free software and open
source : a free software is by definition open source, but the
reciprocal isn't inevitably true. For example, the Android operating
system for mobile phone is open source, everybody can consult its
source code and use it as an inspiration but some part of it are still
proprietary and thus it does not strictly follow the Free Software
Foundation guidelines \cite{and}. It is also important to understant
that free software isn't necesarily free of charge; the \textit{free}
of \textit{free software} rather conveys an idea of freedom and
sharing. As Richard Stallman wrote, it's ``\textit{free} as in
\textit{free speech}, not as in \textit{free beer}'' \cite{free}.

\subsection{A bit of history}

Free software is not a new trend by any means. In fact, free software
was the first distribution model for softwares, even if it didn't have
this name yet. In the 50s, when computers were huge contraptions
reserved to laboratories, notably academic ones, the sharing of source
code was the norm, as the common goal was the discovery of new
knowledge and the advancement of science. Two decades later, when
computers started to appear in corporate environments and later in
homes, softwares rapidly became commercial products. Even if free
software remains relatively unnoticeable compared to its commercial
counterpart, notably because of a lack of advertising, there recently
have been a rise in such projects, with the help of code sharing
initiatives and the participation of several large and influential
companies.

\section{The theory: Is free software viable?}

\subsection{Benefits and drawbacks}

We are going to analyze the advantages of disadvantages of free
software from the perspective of a software company.

Firstly, a company delivering free softwares automatically benefits
from a boost in reputation, as with proprietary software and
closedness come a sense of secret and users are barred from seeing
what is really happening behind the scene. Many companies, Microsoft
for example, are labeled as evil because they favour proprietary
software. An other advantage is that FS prompts trust from users. As
the source code is available, if they have a doubt about the security
of their banking data for example, or the way their password is
encrypted, they can check by themselves if these informations are
tighlty secured or at the reach of any hacker. These two advantages
are related to the users, and they obviously have a positive impact on
a software company as \textit{happy users} means \textit{more users}.

There also are financial advantages that come with free software. As
openness is admitted, there is no need in patenting softwares.  In the
same way, less money will be spent in potential legal conflicts
\cite{afses}. This does not mean that free software garantees legal
security but due to less restrictive licences, lawsuits are less
enclined to happen.

Of course, a major advantage of free software is user
contribution. Any programmer around the world can fix any bug he
stumbles upon and even add missing features, and all that on a
voluntary basis. If we put aside the financial savings that such a
practice permits, the major upside comes from the quality boost that a
well-controled software could gain as users are well-placed to know
what is wrong with a given software and thus they know how to improve
it.

Free software is not perfect and suffers some disadvantages. The
shortcoming of all the previous facts is that they only relate to
experimented users, familiar with the computer and its world, whereas
laymans will prefer a less intimidating solution. We can see here one
of the cons of free software. Some unexperimented users will choose a
proprietary solution because a bigger price tag means more quality.
For example, a user having to pick a new operating operating system
and being suggested Ubuntu, a free OS costing zero dollar, and Windows
7, a proprietary one costing around 200 dollar, is more likely to
chose Windows because a higher price gives the idea that the expensive
product can only be better.  This psychological effect, associated
with the public lack of knowledge about free software, is a
significant brake to the expansion of free software. The main issue
that large software companies reproach to this model is that it does
not permit them to hide some parts of a software, especially the more
innovative ones. The real matter is that their code is visible to
everyone, especially competitors. This is the reason why a number of
software corporations choose proprietary software over free software:
they would have to share their most ingenious algorithms and
revolutionizing features.

Free software seems to be a smart choice, given the savings and user
participation it permits. But we have to keep in mind that the goal of
a company is to generate revenue and, if possible, profits. Of course
the obvious solution is to sell the software (contrary to popular
beliefs, free software can be sold and Richard Stallman even encourage
it \cite{sell}). We present here three different businness models
compatible with free software.

\subsection{Possible business models}

Companies such as Ubuntu use a support business model. Its main
product, the Ubuntu operating system, is free of charge. It however
needs infrastructures to be distributed to its 20 million users and
money to pay its 400 employees. Browsing through the Ubuntu website,
we remark several possible sources of income. The more visible one is
an online merchandise store which sells mugs and wearable items
sporting the Ubuntu logo but this store is a mere marketing tool and
cannot guarantee the financial security of the company.  No, its true
business model is to freely give the product and to sell the
knowledge. Unbuntu staff comprehends the developer of its products but
also a lot of trainers, learning to users the ins and outs of Ubuntu
and helping other corporations to deploy Ubuntu. This kind of services
is not targetted to individuals but rather for corporations wanting to
install Ubuntu on all their computers in order to save money.

The Mozilla Corporation has chosen another path. Its main product, the
Firefox internet browser, is financed by donations but this part of
their revenue only accounts for about 5\% of their total incomes. The
real revenue source is a partnership with Google. In the interface of
Firefox, a small text field can be seen at the top of the window. Any
words that a user will put there is to be processed through the Google
search engine. There are other available search engines, but Google is
used by default. This choice is not random and actually comes from a
contract between Mozilla and the search giant. This partnership
granted Mozilla with 84\% of their annual revenue of approximately
103M dollar in 2009 \cite{moz}.

Chromium, the internet browser from Google, uses yet another business
model. Chromium started as Chrome, an open source browser which
contained a small percentage of proprietary modules. Following the
community demands, Google forked the project and created Chromium, the
free counterpart to Chrome. An interesting fact about Chromium and
Chrome is that they generate zero revenue. This is due to the fact
that Chrome is not Google's real product: Google sells targetted
advertising space and display textual ads related to the search
results of an user. By offering a fast and free internet browser to
users, they let people access their true source of revenue more easily
\cite{shuttle}.

\section{The pratice: What is really happening}

Each one of the largest software companies publicly endorses free
software. They all possess one or several pages on their websites to
explain how they are dedicated to the free software movement and how
they help it move forward. The subtlety being that their definitions
differ greatly. Generally, we can cast software businesses in three
categories.

First, there are the reluctant ones, like Microsoft \cite{woss}. Even
if they remorselessly display a public attachment to openness, few are
their projects really following the FSF guidelines. At best, they open
and document parts of their software to permit operability with other
softwares, potentially free. In general, they don't hesitate to
inflict lawsuits to free software equivalents of their products.

Then we have the conflicted ones. We can cite Apple and Oracle as
examples. Their products are typically based on free softwares and
they participate to the global free software effort while defending
aggressively their core products. Apple is a good example because
while it suffers from a bad reputation in the open source world due to
the closedness of their operating systems and their hardware, they are
surprisingly active behind the scene. They are for example the main
contributors to the Webkit project, a rendering engine used in their
Safari browser, but also in Chrome since it was then released as a
free software.

Lastly, some companies chose to embrace the free software
movement. Ubuntu and Red Hat respectively offer and sell free software
and sustain their financial health by selling support and
services. Intel and Google create free softwares related to their main
products in order to create a sane and accessible software
environement around them. More precisely, Intel is an hardware
manufacturer, famous for its central processing units, but all the
tools that let users tweak and monitor their hardware are given as
free software.

\section{Conclusion}

We cannot realistically expect free software to fully overthrow
proprietary software as some large companies are still relentlessly
reluctant to give a try to this model.  They chose this path not
because they are corporate evil masterminds, as free software
activists tend to caricature them, but rather because they refuse to
put their innovation at the sight of all, notably at the sight of
competitors.

Even if believing that free software will become the leading
distribution model remains utopic, it is important to note that a
growing number of free projects comes to light and that several large
companies are following this trend by involving more and more users in
the making of their softwares.

\section*{Glossary}

\subsection*{Free software}

{\color{bg}Software that everyone is free to copy, redistribute and
  modify. That implies free software must be available as source code,
  hence ``free open source software'' - ``FOSS''. It is usually also free
  of charge, though anyone can sell free software so long as they
  don't impose any new restrictions on its redistribution or use
  \textemdash $\;$ dictionnary.reference.com }

Logiciel que chacun est libre de copier, redistribuer et
modifier. Cela signifie qu'un logiciel libre doit être disponible sous
forme de code source, d'o\`u l'appellation de \textit{logiciel libre
  et open source}. Il est habituellement gratuit, cependant n'importe
qui peut vendre un logiciel libre tant qu'aucune nouvelle
restriction d'utilisation et de redistribution ne sont imposées.

\subsection*{Open source}

{\color{bg}Generically, open source refers to a program in which the
  source code is available to the general public for use and/or
  modification from its original design free of charge \textemdash
  $\;$ Webopedia.com }

Généralement, le terme \textit{open source} fait référence à un
logiciel donc le code source est délivré au public pour l'utiliser
et/ou le modifier et ce, de façon gratuite.

\subsection*{Proprietary software}

{\color{bg}Refers to any computer software that has restrictions on
  any combination of the usage, modification, copying or distributing
  modified versions of the software. Proprietary software may also be
  called closed-source software \textemdash $\;$ Webopedia.com }

Fait référence à tout logiciel informatique soumis à des restrictions
quant à son usage, sa modification, la copie ou la distribution de
versions modifiées ou toute combinaison de ces actions. Les logiciels
propriétaires peuvent aussi être appelés \textit{closed source} (en
opposition à \textit{open source}).

\subsection*{Operating system}

{\color{bg}Software that controls the operation of a computer and
  directs the processing of programs (as by assigning storage space in
  memory and controlling input and output functions) \textemdash $\;$
  Merriam-Webster }

Logiciel contrôlant les opérations d'un ordinateur et dirigeant les
calculs nécessaires aux programmes (en assignant de l'espace de
stockage en mémoire et en contrôlant les fonctions d'entrée et de
sortie par exemple).

\subsection*{Browser}

{\color{bg}A computer program used for accessing sites or information
  on a network (as the World Wide Web) \textemdash $\;$
  Merriam-Webster }

Un programme informatique utilisé pour accéder à des sites ou des
informations sur un réseau (le réseau Internet par exemple).

\bibliographystyle{plain}
\bibliography{sources}

\end{document}
