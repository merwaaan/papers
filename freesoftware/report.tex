\documentclass[12pt]{article}

\usepackage[utf8]{inputenc}
\usepackage{url}

\title{Are large software companies bound to adopt the free software model?}
\author{Merwan Achibet\\Université du Havre}
\date{}

\begin{document}

\maketitle
\newpage

\tableofcontents
\newpage

\section*{Abstract}

abstract

\section*{Keywords}

keywords

\newpage

\section*{Introduction}

\section{What is free software?}

Before pondering about the interest that large software companies
might have for the free software model, we have to clearly define
it. The concept has a large spectrum of significance and definitions
are many. Nevertheless, they all share the same basis
\cite{sers}. Firstly, the intrinsec quality of this kind of software
comes from its openness: its source code is available for all, and
above all, editable by all. There are thus no restriction on the
amount of hacking that such a software can endure, which is a good
thing, as the next section will demonstrate. Secondly, the available
source code is candidate to redistribution, with or without
modification. Lastly, a free software can be used by all and for any
means. A differencation must be made between free software and open
source. A free software is by definition open-source, but the
reciprocal isn't inevitably true. For example, the Android operating
system for mobile phone is open-source, everybody can consult its
source code and use it as an inspiration. But one does not have the
right to entirely copy this code and rebrand it as his own. The
definition of copy is fuzzy here. What differenciate a full copy from
a largely inspired work? Of course, free software is free, in the
sense that it doesn't cost any money to the user, but the free of the
expression carries a sense of freedom and liberal philosophy, not only
the monetary aspect.

Founded by Richard Stallman, a unique computer scientist, the free
software movement is rooted in deep philosophical considerations which
often get away from the more technical aspects. It was firstly mocked
by large software companies, judged inneficient and idealist. Present

\section{From the point of view of software companies}

With free software come advantages, some related to the user base of
the company, some related to its economic viability.

\subsection{Impact on the user base}

A company delivering free softwares automatically benefits from a
boost in reputation, as with proprietary software and closedness come
a sense of secret. Many companies, Microsoft for example, are labeled
as evil because of their model. An other advantage is the openness of
the source code, as experimented users can check for themselves the
quality of the product by exploring its code. Moreover, a lot of
questions stem from confidentiality interrogations. With proprietary
software, a user can ask himself how his passwords are protected and
if his credit card number is stored somewhere, at the reach of
pirates. With free software, he can check that himself. And of course,
the money aspect is a major advantage. When comparing two solutions, a
user will easily lean toward the less pricy one, ideally, the free
one!

The shortcoming of all these advantages is that they only concern
experimented users, familiar with the computer and its world, whereas
laymans will prefer a less intimidating solution. We can see here one
of the downfalls of the free, free as in no money needed,
software. Some unexperimented users will choose a proprietary solution
because a bigger price tag means more quality. How can this free
stuff be as good as the 200\$ one? This psychological effect,
associated with the public lack of knowledge about free software, is
the main brake to free software taking off. Moreover, a free software
is built from others free softwares, contrary to proprietary softwares
which pay licences to use some proprietary component, and the average
computer user has a Windows operating system and a typical suite of
softwares, often proprietary. Making him change his software suite and
get out of his confort zone is no easy task.

\subsection{Impact on economy}

As a free software can be copied and reused by anybody, there is no
use in patenting, nor in budgeting potential legal
conflicts. Financially speaking, free software shows to be the
smartest choice. As stated previously, an open software is built upon
blocks of existing free softwares whereas in the proprietary world,
the blocks are often proprietary too and as such, they are to be
paid. From the other perspective, our free software could also be used
as a building block, and its non-existent price could promote its
regular, maybe even its use as an industry standard. Let's not forget
an important building force: user contribution. Of course, any user
can copy the free software and label it as its own, but that's happily
not the mentality os this scene. Instead, a user will often contribute
to a project by the means of personal bug corrections or feature
additions. All that being free of any charge, and made by people who
use the software, who are therefore in first line to judge what are
the weaknesses of a software.

Openness has a price, though, and the fact that the machinery of a
software is fully available implies that even the bleeding-edge
features, the smartest algorithms and the revolutionnary ideas are at
the reach of all, especially the concurrence. A line has to be drawn
here, this is where the company has to make choice. Will it contribute
to a joint progress or will it prefer to hide its discoveries and make
money? And of course

\section{Is it viable?}

One the main characteristics of free software being its price,
companies obviously need to switch to a more adequate business model
to stay afloat. We have to keep in mind that the goal of a company is
to make money, making free software or not, and this model must be
economically viable.

Companies such as Ubuntu use a support business model. Ther main
product, the Ubuntu operating system, is free. They even sent millions
of free CDs containing it for a while. Browsing through the foundation
website, we remark seeral possible sources of income. The more visible
is an online merchandise store which sells mugs and wearable items
sporting the Ubuntu logo but this store is more of a marketing tool
than a true mean of revenue. No, its true business model is to freely
give the tools and to sell the knowledge. Unbuntu staff comprehends
the developer of its products but also a lot of trainers, learning to
users the ins and outs of Ubuntu. This kind of technical support is
not meant for simple individual users but rather for corporations
wanting to install Ubuntu on all their computers and to train
employees.

The Mozilla Foundation has chosen another path, its main product, the
Firefox internet browset, is financed by donations but this part of
their revenue only accounts for about 5\% of the total. The real
revenue source is a partnership with Google. In Firefox's interface, a
small search field can be seen at the top of the window. Any words the
user will input there will be processed through Google Search,
generating ad revenue to the search engine, a part of which is
reversed to Mozilla. Stats

Another browser example is Chrome, the internet browser from Google.

\section*{Conclusion}

\section*{Glossary}

\begin{description}
  \item[Free software]{a}
  \item[Open source]{a}
  \item[GNU]{a}
  \item[Free Software Foundation]{a}
  \item[Richard Stallman]{a}
\end{description}

\section*{Sources}

\bibliographystyle{plain}
\bibliography{sources}

\end{document}
