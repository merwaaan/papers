\documentclass[12pt]{article}

\usepackage[utf8]{inputenc}
\usepackage{url}

\renewcommand{\baselinestretch}{1.5}

\title{Are Large Software Companies Bound to Adopt the Free Software Model?}
\author{Merwan Achibet\\Université du Havre}
\date{}

\begin{document}

\maketitle
\newpage

\tableofcontents
\newpage

\section*{Abstract}

With the ever-increasing accessibility to computers, large software companies products typically are
proprietary softwares, but with the growth of the free and open-source movement,
we have to ask ourselves if this situation is bound to change. An overview of the advantages of FS and a presentation of several business models compatible with FS shows that it has the capability to attract software companies. Even if FS and PS are meant to cohabit without one model overthrowing the other, we observe a clear increase in the number of companies involved with free projects and we explain that the main brake to this growth comes from the intrinsec  openness of FS which does not let companies hide their innovation from competitors.

\section*{Keywords}

Free software, open source, proprietary software, software companies

\newpage

\section*{Introduction}

TODO

\section{Context}

Before pondering about the interests that large software companies
might have for the free software model, we have to clearly define
it. The concept has a large spectrum of significance and definitions
are many, often separated by subtle differences. Nevertheless, they
all share the same basis \cite{sers}. Firstly, the intrinsec quality
of this kind of software comes from its openness: its source code is
available for all, and above all, editable by all. There are thus no
restriction on the amount of hacking that such a software can endure. Secondly,
the available source code is candidate to redistribution, as long as
the new software perpetuate the original licence. A differencation must
be made between free software and open source : a free software is by
definition open-source, but the reciprocal isn't inevitably true. For
example, the Android operating system for mobile phone is open-source,
everybody can consult its source code and use it as an
inspiration but some part of it are still proprietary and thus it does not strictly follow the
Free Software Foundation guidelines. It is also important to understant that free software isn't
necesarily free of charge. It rather conveys an idea of freedom and sharing.

Free software is not a new trend by any means. In fact, free software
was the first distribution model for softwares, even if it didn't have
this name yet. In the 50s, when computers where huge contraptions
reserved to laboratories, notably academic ones, the sharing of source
code was the norm, as the common goal was the discovery of new knowledge and the
advancement of science. Two decades later, when computers started to
appear in corporate environments and later in homes, softwares rapidly
became commercial products. Even if free software remains relatively unnoticeable
compared to its commercial counterpart, there recently have been a
rise in such projects, with the help of code sharing initiatives and
the participation of several large companies.

\section{The theory: Is free software viable?}

We are going to analyze the advantages of disadvantages of free software from the perspective of 
a software company.
A company delivering free softwares automatically benefits from a
boost in reputation, as with proprietary software and closedness come
a sense of secret. Many companies, Microsoft for example, are labeled
as evil because of this model. An other advantage is that FS prompts
trust from users. If they have a doubt about the security of their
banking data dor example, or the way their password is encrypted, they
can check by themselves. With free software, openness is
admitted. Thus there is less need in patenting softwares,
nor in budgeting potential legal conflicts. Of course, a major
advantage of free software is user contribution. Any programmer around the world
can fix any bug he stumbles upon and even add missing features, and
all that on a voluntary basis. If we put aside the financial savings
that such a practice permits, the major upside comes from the quality
boost that a well-controled software could gain as the users
are well-placed to know what is wrong with a software and thus
how to improve it.

The shortcoming of all these advantages is that they only concern
experimented users, familiar with the computer and its world, whereas
laymans will prefer a less intimidating solution. We can see here one
of the downfalls of the free \textemdash free as in free in charge
\textemdash software. Some unexperimented users will choose a
proprietary solution because a bigger price tag means more quality.
For example, a user having to pick a new operating operating system and being suggested Ubuntu, a free OS costing zero dollar, and Windows 7, a proprietary one costing around $200, is more likely to chose Windows because a higher price gives the idea
This psychological effect, associated with the public lack of knowledge
about free software, is the main brake to free software
expansion. The main problem that large software companies reproach to
this model is that it does not permit them to hide some parts of a
software, especially the more innovative ones. The real problem is that their
code is visible to everyone, especially competitors. This is the reason why
a number of software corporations choose PS over FS: they would have
to share their most ingenious algorithms and revolutionizing ideas.

Free software seems to be a smart choice, given the savings and user
participation it furnishes. But we have to keep in mind that the goal
of a company is to generate revenue and preferably, earnings. Of
course the obvious solution is to sell the software, as contrary to
popular beliefs, free software can be sold and Richard Stallman even
encourage it CITATION NEEDED. We present three different businness
models compatible with free software.

Companies such as Ubuntu use a support business model. Ther main
product, the Ubuntu operating system, is free of charge. They however
need infrastructures to distribute it to their 20 million users and money to pay their
400 employees. Browsing through the Ubuntu website, we remark several
possible sources of income. The more visible is an online merchandise
store which sells mugs and wearable items sporting the Ubuntu logo but
this store is a mere marketing tool and cannot guarantee financial security. 
No, its true business model is to freely give the tools and
to sell the knowledge. Unbuntu staff comprehends the developer of its
products but also a lot of trainers, learning to users the ins and
outs of Ubuntu and helping other corporations to deploy Ubuntu. This
kind of services is not targetted to individuals but rather for
corporations wanting to install Ubuntu on all their computers in order to save money.

The Mozilla Corporation has chosen another path. Its main product, the
Firefox internet browser, is financed by donations but this part of
their revenue only accounts for about 5\% of their total incomes. The
real revenue source is a partnership with Google. In Firefox's
interface, a small search field can be seen at the top of the
window. Any words that a user will put there is to be processed through
the Google search engine. There are other available search engines, but by
default Google is used. This choice is not random and actually comes
from a contract between Mozilla and the search giant. This partnership
grants Mozilla with 84\% of their annual revenue (approximately $103M in 2010 CITATION NEEDED).

Another browser example is Chromium, the internet browser from
Google. Chromium started as Chrome, an open-source browser which
contained a small part of proprietary modules. Following the community
demands, Google forked the project and created Chromium, the
free counterpart to Chrome. An interesting fact about Chromium and Chrome is that
they generate zero revenue. This is due to the fact that Chrome is not
Google's real product: Google sells targetted advertising space and
display textual ads related to the search results of an user. By
offering a fast and free internet browser to users, they let people access their true source
of revenue more easily.

\section{The pratice: What is really happening}

Each one of the largest software companies publicly endorses free
software. They all possess one or several pages on their websites to
explain how they are dedicated to the free software movement in this
era of global sharing. The subtlety being that their definitions
differ greatly. We can cast software businesses in three categories.

First, there are the reluctant ones, like Microsoft. Even if they
display a public attachment to openness, few are their projects really
following the FSF guidelines. At best, one could hope that they would
document their software to permit operability with other softwares,
potentially free. In general, they don't hesitate to inflict lawsuits
to free software equivalents of their products.

Then we have the conflicted ones. We can cite Apple and Oracle as
examples. Their products are typically based on free softwares and
they participate to the glocal FS effort while defending aggressively
their core products. Apple is a good example because while it suffers
from a bad reputation in the open source world due to the closedness
of their OSs and their hardware, they are surprisingly active behind the scene. They are for example the
main contributors to the Webkit project, a rendering engine used in their Safari browser, but also in Chrome since it was released
as a free software.

Lastly, some companies chose to embrace the free software
movement. Ubuntu and Red Hat respectively offer and sell free software
and sustain their financial health by selling support and
services. Intel and Google create free softwares related to their main
products in order to create a sane and accessible software
environement around them.

\section*{Conclusion}

We cannot realistically expect free software to fully overthrow proprietary software
as some large companies are still agressively reluctant to give a try to this model.
They chose this path not because of an imaginary corporate evil imagery but rather because
they refuse to display their innovation at the sight of all, notably at the sight of competitors.

Even if believing that FS will become remains utopic, it is important to note
that a growing number of free projects come to light and that several large companies
are following this trend by involving users in the making of their softwares.

\section*{Glossary}

\begin{description}
  \item[Free software]{An open software, usable, modifiable and redistributable by all}
  \item[Open source]{An open software, everyone can read its source code but it can still be subject to copyright restrictions}
  \item[Proprietary software]{A closed software whose usage and accessibility to show the source code is restricted}
  \item[Free Software Foundation]{Organization founded by Richard Stallman whose goal is to promote the use and creation of free software}
\end{description}

\section*{Sources}

\bibliographystyle{plain}
\bibliography{sources}

\end{document}
