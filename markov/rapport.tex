\documentclass[12pt]{article}

\usepackage[utf8]{inputenc}
\usepackage[francais]{babel}
\usepackage{amsmath}
\usepackage{float}
\usepackage{tikz}
\usepackage{url}

\usetikzlibrary{arrows, positioning}

\title{Chaînes de Markov}
\author{Merwan Achibet\\Université du Havre}
\date{}

\begin{document}

\maketitle

\section{Concept}

\subsection{Généralités}

Une chaîne de Markov est un processus aléatoire à états pour lequel la
propriété de Markov est vérifiée.

\subsubsection{Processus aléatoire à états}

Un processus aléatoire à états prend la forme d'un vecteur $X$
représentant une liste d'états parcourus, chaque état possible
appartenant à un ensemble d'états $S = \{x_0, \dots, x_n \}$. \`A
intervalle régulier (on parle alors de processus temporellement
homogène), on peut soit rester dans le même état soit passer à un
nouvel état.

Un exemple classique pour illustrer le principe des chaînes de Markov
est celui de la marche aléatoire \cite{ocone} : on veut modéliser
discrètement le déplacement d'un individu dans un environnement à une
dimension, typiquement de gauche à droite. L'ensemble des états peut
ici correspondre à l'indice de la position courante, ainsi $S = \{1,
2, 3, 4, 5\}$ pour 5 positions différentes. Prenons arbitrairement
$X_0 = 2$; l'individu se trouve donc au départ sur la seconde position
à partir de la gauche. \`A chaque étape, l'individu doit se déplacer
vers une position voisine, ce qui revient à lui proposer deux choix
(gauche ou droite), sauf lorsqu'il est à l'une des extrêmités de
l'environnement, cas dans lequel il n'a qu'un seul choix (il va à
droite s'il est à l'extrêmité gauche, et \textit{vice versa}).

\begin{figure}[H]
\centering

\begin{tikzpicture}[node distance=1cm, auto]

  \node[circle, draw] (1) {1};
  \node[circle, draw, right=of 1] (2) {2};
  \node[circle, draw, right=of 2] (3) {3};
  \node[circle, draw, right=of 3] (4) {4};
  \node[circle, draw, right=of 4] (5) {5};

  \draw[->, thick] (1) to[bend left=60] node[midway] {1} (2);
  \draw[->, thick] (2) to[bend left=60] node[midway] {0.5} (3);
  \draw[->, thick] (3) to[bend left=60] node[midway] {0.5} (4);
  \draw[->, thick] (4) to[bend left=60] node[midway] {0.5} (5);

  \draw[->, thick] (5) to[bend left=60] node[midway] {1} (4);
  \draw[->, thick] (4) to[bend left=60] node[midway] {0.5} (3);
  \draw[->, thick] (3) to[bend left=60] node[midway] {0.5} (2);
  \draw[->, thick] (2) to[bend left=60] node[midway] {0.5} (1);

\end{tikzpicture}

\caption{Représentation du problème de la marche aléatoire.}

\end{figure}

Usuellement, on se base sur une matrice de transition $T$ pour décrire
les probabilités de passer d'un état précis à un autre. Pour cet
exemple, la matrice de transition est la suivante :

$$
T = \bordermatrix{
    & 1 & 2 & 3 & 4 & 5 \cr
  1 & 0 & 1 & 0 & 0 & 0 \cr
  2 & 0.5 & 0 & 0.5 & 0 & 0 \cr
  3 & 0 & 0.5 & 0 & 0.5 & 0 \cr
  4 & 0 & 0 & 0.5 & 0 & 0.5 \cr
  5 & 0 & 0 & 0 & 1 & 0 \cr
}
$$
\vspace{0.5cm}

On remarque que tous les $T_{ii}$ avec $i \in \{1, \dots, 5\}$ valent
0, car l'individu ne peut pas rester à la même position pendant deux
étapes successives. La probabilité de rester dans le même état est
donc nulle. De plus les probabilités de passer à un état représentant
une position non voisine de la position actuelle sont elles aussi
nulles, car l'individu est limité à des déplacements vers les
positions adjacentes. Bien sûr, $\forall i \in \{1, \dots, 5\}$,
$\sum_{j=1}^5 T_{ij} = 1$

\`A partir de $T$, et si l'on veut prévoir la position de l'individu à
la deuxième étape (on cherche donc $X_1$), on remarque que seules les
transitions $2 \rightarrow 1$ et $2 \rightarrow 3$ sont
envisageables. Si l'on dispose d'un générateur de valeurs aléatoires,
on peut alors simuler la décision de l'individu et choisir en
conséquence son état à l'étape suivante. Puisque $P(X_1 = 1 | X_0 = 2)
= 0.5$ et que $P(X_1 = 3 | X_0 = 2) = 0.5$, l'individu a une chance
sur deux d'aller vers la position 1 et une chance sur deux d'aller
vers la position 3.

\subsubsection{Propriété de Markov}

Maintenant que la notion de processus aléatoire à états est présentée,
il faut préciser la propriété de Markov à laquelle une chaîne de
Markov se doit d'obéir.

Concrètement, cette propriété énonce que dans le cadre d'un processus
aléatoire à états, l'état futur dépend uniquement de l'état présent et
non de tous les états précédents.

Plus formellement, si le processus en question a atteint sa
$t$\textsuperscript{ème} étape :

$$
P(X_{t+1} | X_{t}, X_{t-1}, \dots, X_0) = P(X_{t+1} | X_t)
$$

Autrement dit, seul l'état courant du système nous importe. Cette
propriété est intéressante, car dans l'exemple de la marche aléatoire,
un processus à états basique peut se servir de la matrice de
transition pour déterminer $X_1$ à partir de $X_0$ alors qu'une chaîne
de Markov peut s'en servir déterminer $X_{t+1}$ à partir de n'importe
quel $X_t$.

\subsection{Interêts pour la modélisation}

Un système peut être modélisé par un large choix de formalismes. Parmi
les modélisations envisageables, on pense notamment à la modélisation
individu-centrée, ou bien à l'application de lois logistiques ou à
l'utilisation de systèmes d'équations. Un avantage des chaînes de
Markov pour réprésenter un problème est l'éventail de possibilités de
quantification et d'analyse qu'elles offrent \cite{izq}.

Contrairement à une modélisation individu-centrée, de laquelle on ne
peut extraire des analyses qu'à partir des résultats opérationnels,
plusieurs aspects d'un modèle de type chaîne de Markov peuvent être
évalués avant toute éxecution, à partir de la matrice de
transition. On peut évidemment prévoir l'état suivant d'une chaîne
mais aussi faire des prévisions sur l'état qui sera atteint au bout de
$x$ étapes. Il est de plus possible d'étudier la périodicité d'une
chaîne (si le modèle admet des cycles dans sa progression) et sa
récurrence (le nombre de fois qu'un état particulier est pris).

\subsubsection{Prévision}

Comme énoncé précédemment, la propriété de Markov permet de prévoir
l'état suivant d'une chaîne quel que soit le numéro de l'étape
actuelle. Autrement dit, que l'on en soit à l'étape initial ou cent
étapes plus tard, on peut évaluer la possibilité de prendre un certain
état à l'étape suivante.

En conséquence, il est aussi possible de calculer la possibilité
d'atteindre un état $j$ à partir d'un état $i$ au bout d'un certain
nombre d'étapes. Reprenons l'exemple de la marche aléatoire, pour
lequel $X_0 = 2$. On veut déterminer la probabilité que l'individu
soit encore à la position 2 à la 3\textsuperscript{ème} étape, soit
deux étapes plus tard. Autrement dit, on veut calculer $P(X_2 = 2 |
X_0 = 2)$. On note cette probabilité $p^{(2)}_{2,2}$ \cite{green}.

On peut représenter conceptuellement toutes les issues possibles de
cette situation par un arbre dans lequel chaque niveau correpond à une
étape (figure \ref{arbre}). On remarque que dans deux cas sur trois,
on atteint l'état 2 à partir de l'état 2.

\begin{figure}[H]
\centering

\begin{tikzpicture}[level/.style={sibling distance=60mm/#1}]

  \tikzstyle{n}=[circle, draw]
  \tikzstyle{p}=[rectangle, draw=none, fill=gray!30]

  \node[n] (a) {$2$}
  child {
    node[n] (b) {$1$}
    child {
      node[n] (c) {$2$}
    }
  }
  child {
    node [circle,draw] (d) {$3$}
    child {
      node[n] (e) {$2$}
    }
    child {
      node[n] (f) {$4$}
      child[grow=right] {
        node {$t_2$} edge from parent[draw=none]
        child[grow=up] {
          node {$t_1$} edge from parent[draw=none]
          child[grow=up] {
            node {$t_0$} edge from parent[draw=none]
          }
        }
      }
    }
  };

  \draw (a) to node[p, midway] {0.5} (b);
  \draw (b) to node[p, midway] {1} (c);
  \draw (a) to node[p, midway] {0.5} (d);
  \draw (d) to node[p, midway] {0.5} (e);
  \draw (d) to node[p, midway] {0.5} (f);

\end{tikzpicture}

\caption{Arbre représentant les états atteignables après deux étapes
  et en partant de la position 2.}
\label{arbre}

\end{figure}

Pour calculer manuellement $p^{(2)}_{2,2}$, on évalue la probabilité
de chaque branche valide par rapport à toutes les issues possibles :

$$
p^{(2)}_{2,2}
= \frac{p_{2,1} p_{1,2} + p_{2,3} p_{3,2}}{p_{2,1} p_{1,2} + p_{2,3} p_{3,2} + p_{2,3} p_{3,4}}
= \frac{0.5 \times 1 + 0.5 \times 0.5}{0.5 \times 1 + 0.5 \times 0.5 + 0.5 \times 0.5}
= 0.75
$$
\vspace{0.5cm}

Ce calcul, ici simple, se révélera plus fastidieux lorsque la
différence temporelle entre les deux étapes considérées s'agrandira,
car le nombre de branches augmentera exponentiellement.

Heureusement, ce même résultat peut être obtenu en exploitant la
matrice de transition. Plus généralement, la probabilité d'être en
l'état $j$ à partir de l'état $i$ après $n$ étape est $p^{(n)}_{ij} =
T^n_{ij}$ \cite{snell}.

$$
T^2 = \bordermatrix{
    & 1 & 2 & 3 & 4 & 5 \cr
  1 & 0.5 & 0 & 0.5 & 0 & 0 \cr
  2 & 0 & 0.75 & 0 & 0.25 & 0 \cr
  3 & 0.25 & 0 & 0.5 & 0 & 0.25 \cr
  4 & 0 & 0.25 & 0 & 0.75 & 0 \cr
  5 & 0 & 0 & 0.5 & 0 & 0.5 \cr
}
$$
\vspace{0.5cm}

Dans ce cas, on retrouve bien $T^2_{2,2} = 0.75$. On note que cette
opération fournit les probabilités de transition projetées vers tous
les états et à partir de tous les états, contrairement au calcul
manuel utilisé plus tôt.

\subsubsection{Absorption}

Un état absorbant est un état duquel on ne peut pas sortir une fois
atteint. Une chaîne de Markov absorbante est une chaîne de Markov
possédant au moins un état absorbant. Ce type de chaîne de Markov
autorise plusieurs facilités d'analyse.

\cite{snell} utilise une variante de la marche aléatoire pour
illustrer le principe des chaînes de Markov absorbantes : l'individu a
un peu trop bu et la position 1 correspond à son domicile tandis que
la position 5 correspond à un bar. S'il atteint un de ces états, il
n'en sort plus, ce sont les états absorbants. Lorsqu'un état absorbant
est atteint, on dit que la chaîne est absorbée.

La nouvelle matrice de transition est :

$$
T = \bordermatrix{
    & 1 & 2 & 3 & 4 & 5 \cr
  1 & 1 & 0 & 0 & 0 & 0 \cr
  2 & 0.5 & 0 & 0.5 & 0 & 0 \cr
  3 & 0 & 0.5 & 0 & 0.5 & 0 \cr
  4 & 0 & 0 & 0.5 & 0 & 0.5 \cr
  5 & 0 & 0 & 0 & 0 & 1 \cr
}
$$
\vspace{0.5cm}

Grâce au principe des chaînes de Markov et à des opérations de base
sur la matrice de transition $T$, si l'on dispose d'une chaîne
absorbante, on peut répondre à des questions telles que :

\begin{enumerate}
  \item{Quelle est la probabilité pour que l'individu finisse par être
    bloqué ?}
  \item{Au bout de combien de déplacements l'individu sera-t-il
    bloqué ?}
  \item{Combien de fois l'individu va-t-il passer par chaque position ?}
\end{enumerate}

Pour répondre à ces questions, il est nécessaire de passer la matrice
de transition sous forme canonique. Ce passage s'opère par permutation
des lignes et des colonnes de la matrice de transition de façon à ce
que les entrées des états absorbants valant 1 forment une matrice
identité de taille $n \times n$ avec $n$ le nombre d'états absorbants.

$$
T = \bordermatrix{
    & 2 & 3 & 4 & 1 & 5 \cr
  2 & \color{red}0 & \color{red}0.5 & \color{red}0 & 0.5 & 0 \cr
  3 & \color{red}0.5 & \color{red}0 & \color{red}0.5 & 0 & 0 \cr
  4 & \color{red}0 & \color{red}0.5 & \color{red}0 & 0 & 0.5 \cr
  1 & 0 & 0 & 0 & \color{blue}1 & \color{blue}0 \cr
  5 & 0 & 0 & 0 & \color{blue}0 & \color{blue}1 \cr
}
=
\begin{pmatrix}
  \color{red}Q & * \\
  0 & \color{blue}I
\end{pmatrix}
$$
\vspace{0.5cm}

On décompose donc $T$ en :

$$
Q =
\begin{pmatrix}
  0 & 0.5 & 0 \\
  0.5 & 0 & 0.5 \\
  0 & 0.5 & 0
\end{pmatrix}
\;\;\;\;\;
I =
\begin{pmatrix}
  1 & 0 \\
  0 & 1
\end{pmatrix}
$$
\vspace{0.5cm}

Si l'on souhaite par exemple estimer le nombre de passages par chaque
état avant l'absorption de la chaîne, on calcule la matrice
fondamentale $N$ telle que :

$$
N = (I - Q)^{-1}
$$

Chaque ligne de $N$ correspond au nombre de passage attendu par chaque
autre état en ayant pris l'état de la ligne comme point de
départ. Dans le cas de la marche aléatoire alcoolisée, on prévoit que
l'individu commençant son trajet à la position 3 passera en moyenne
une fois sur la position 2 avant absorption.

$$
N =
\bordermatrix{
  & 2 & 3 & 4 \cr
  2 & 1.5 & 1 & 0.5 \cr
  3 & 1 & 2 & 1 \cr
  4 & 0.5 & 1 & 1.5
}
$$

\section{Application : noms de la mythologie grecque}

Les chaînes de Markov sont couramment utilisées pour générer du texte
de façon aléatoire à partir d'un corpus. Nous allons ainsi implémenter
un générateur de noms inspiré de ceux des protagonistes de la
mythologie grecque.

\subsection{Modèle}

Une chaîne de Markov correspond ici à un nom soit un mot. Un mot est
composé de lettres, chacun correspondant à un état. Ici l'ensemble des
états est $S = \{a, b, c, \dots, z, \xi \}$

\subsection{Implémentation}

\subsubsection{Apprentissage}


\subsubsection{Génération}


\subsection{Résultats}

\bibliographystyle{alpha}
\bibliography{sources}

\end{document}
