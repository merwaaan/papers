\documentclass{beamer}

\usepackage[utf8]{inputenc}
\usepackage[francais]{babel}
\usepackage{graphicx}
\usepackage{tikz}
\usepackage{listings}

\usetheme{Warsaw}

\setbeamertemplate{navigation symbols}{}
\setbeamertemplate{footline}[frame number]

\definecolor{Backdrop}{RGB}{85, 98, 112}
\definecolor{Foreground}{RGB}{238, 238, 238}
\definecolor{Fabulous}{RGB}{210, 111, 127}

\usecolortheme[named=Backdrop]{structure}
\setbeamerfont{title}{series=\bfseries}
\setbeamercolor{title}{bg=Backdrop}
\setbeamercolor{frametitle}{fg=white, bg=Backdrop}
\setbeamercolor{normal text}{fg=Backdrop, bg=white}
\setbeamerfont{alerted text}{series=\bfseries}
\setbeamercolor{alerted text}{fg=Backdrop!120}
\setbeamertemplate{blocks}[rounded][shadow=true]
\setbeamerfont{block title}{series=\bfseries}
\setbeamercolor{block title}{fg=white, bg=Backdrop!105}
\setbeamercolor{block body}{fg=Foreground, bg=Backdrop!80}
\setbeamertemplate{items}[circle]
\setbeamercolor{item}{fg=Fabulous!50}
\setbeamerfont{footline}{size=\small}

\title{Modèle acteur et Scala}
\author{Merwan Achibet}
\institute{Université du Havre}
\date{Vendredi 24 février 2012}

\begin{document}

\maketitle

\section{Le modèle acteur}

\begin{frame}

  \begin{block}{Acteurs}
    Processus concurrent communiquant avec d'autres acteurs par
    échange de messages. Un acteur peut répondre à un message
    asynchrone en créant un nouvel acteur, en envoyant des messages ou
    en changeant de comportement. \cite{haller}
  \end{block}

  \vfill

  Idées conductrices :

  \begin{itemize}
  \item{Tout est acteur}
  \item{Asynchronisme}
  \item{Fault tolerance}
  \end{itemize}

\end{frame}

\begin{frame}

  \begin{block}{Motivations}
    \begin{itemize}
    \item{Parallélisation croissante du matériel}
    \item{Distribution des calculs}
    \item{Un paradigme structuré autour de ces idées}
    \end{itemize}
  \end{block}

  \vfill

  \begin{figure}

    \centering

    \begin{tikzpicture}
    
      \def \n {5}
      \def \radius {2cm}
      \def \margin {8}
    
      \foreach \s in {1,...,\n}
      {
        \node[draw, circle] (\s) at ({360/\n * (\s - 1)}:\radius) {$\s$};
        \draw[->, >=latex] ({360/\n * (\s - 1)+\margin}:\radius) 
        arc ({360/\n * (\s - 1)+\margin}:{360/\n * (\s)-\margin}:\radius);
      }

      \draw (3) edge[->,>=latex,bend right] (1);
      \draw (1) edge[->,>=latex,bend right] (5);

    \end{tikzpicture}

    \end{figure}

\end{frame}

\begin{frame}

  \frametitle{Modèle acteur au sens strict \cite{rajesh}}

  \begin{block}{State encapsulation}
    Aucun partage de donnée hormis les messages.
  \end{block}

  \begin{block}{Safe-messaging}
    Les messages contiennent des copies strictes.
  \end{block}

  \begin{block}{Mobility}
    Le code et l'état d'un agent peuvent se déplacer entre
    processeurs, n\oe uds d'un réseau...
  \end{block}

  \begin{block}{Location transparency}
    Quelle que soit sa position, un agent dispose de la même adresse
    et tout message l'atteindra.
  \end{block}

\end{frame}

\begin{frame}

  \frametitle{JVM et concurrence}

  \begin{columns}

    \begin{column}{0.5\textwidth}
      \begin{block}{Thread-based}
        \begin{itemize}
        \item[+]{Simple}
        \item[-]{Lourd}
        \item[-]{Deadlock}
        \end{itemize}
      \end{block}
    \end{column}
    
    \begin{column}{0.5\textwidth}
      \begin{block}{Event-based}
        \begin{itemize}
        \item[+]{Performant}
        \item[-]{Vite tortueux}
        \end{itemize}
      \end{block}
    \end{column}

  \end{columns}
  
\end{frame}

\section{Scala}

\begin{frame}

  \frametitle{Le langage Scala}

  \begin{block}{Origines}
    \begin{itemize}
    \item{Créé en 2003 à l'EPFL}
    \item{Académique}
    \item{Pragmatique}
    \item{Versatile}
    \end{itemize}
  \end{block}

  \vfill

  \begin{columns}

    \begin{column}{0.5\textwidth}
      %\begin{lstlisting}
      %d
      %\end{lstlisting}
    \end{column}
    
    \begin{column}{0.5\textwidth}
      \begin{figure}
        \centering
        \includegraphics[width=6cm]{scala.png}
      \end{figure}
    \end{column}

  \end{columns}
  
\end{frame}

\begin{frame}
  
  \frametitle{Un langage mixte}

  \begin{columns}

    \begin{column}{0.5\textwidth}
      \begin{block}{Aspect OO}

      \end{block}
    \end{column}
    
    \begin{column}{0.5\textwidth}
      \begin{block}{Aspect fonctionnel}

      \end{block}
    \end{column}

  \end{columns}
  
\end{frame}

\begin{frame}

  

\end{frame}

\bibliographystyle{alpha}
\bibliography{references}

\end{document}
