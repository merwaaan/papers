\documentclass{beamer}

\usepackage[utf8]{inputenc}
\usepackage[francais]{babel}
\usepackage{graphicx}
\usepackage{tikz}
\usepackage{listings}
\usepackage{ucs}

% Scala highlighting definition
% From http://tihlde.org/~eivindw/latex-listings-for-scala/
\lstdefinelanguage{Scala}{
  morekeywords={abstract,case,catch,class,def,%
    do,else,extends,false,final,finally,%
    for,if,implicit,import,match,mixin,%
    new,null,object,override,package,%
    private,protected,requires,return,sealed,%
    super,this,throw,trait,true,try,%
    type,val,var,while,with,yield},
  otherkeywords={=>,<-,<\%,<:,>:,\#,@},
  sensitive=true,
  morecomment=[l]{//},
  morecomment=[n]{/*}{*/},
  morestring=[b]",
  morestring=[b]',
  morestring=[b]"""
}

\lstset{ %
  basicstyle=\tiny,
  extendedchars=\true,
  inputencoding=utf8x,
  language=Scala,                % the language of the code
  showspaces=false,               % show spaces adding particular underscores
  showstringspaces=false,         % underline spaces within strings
  showtabs=false,                 % show tabs within strings adding particular underscores
  tabsize=1,                      % sets default tabsize to 2 spaces
  captionpos=b,                   % sets the caption-position to bottom
  breaklines=true,                % sets automatic line breaking
  breakatwhitespace=false,        % sets if automatic breaks should only happen at whitespace
  keywordstyle=\color{blue},          % keyword style
  commentstyle=\color{dkgreen},       % comment style
  stringstyle=\color{purple},         % string literal style
  escapeinside={\%*}{*)},            % if you want to add a comment within your code
  morekeywords={*,...}               % if you want to add more keywords to the set
}

\usetheme{Warsaw}

\setbeamertemplate{navigation symbols}{}
\setbeamertemplate{footline}[frame number]

\definecolor{Backdrop}{RGB}{85, 98, 112}
\definecolor{Foreground}{RGB}{238, 238, 238}
\definecolor{Fabulous}{RGB}{210, 111, 127}

\usecolortheme[named=Backdrop]{structure}
\setbeamerfont{title}{series=\bfseries}
\setbeamercolor{title}{bg=Backdrop}
\setbeamercolor{frametitle}{fg=white, bg=Backdrop}
\setbeamercolor{normal text}{fg=Backdrop, bg=white}
\setbeamerfont{alerted text}{series=\bfseries}
\setbeamercolor{alerted text}{fg=Backdrop!120}
\setbeamertemplate{blocks}[rounded][shadow=true]
\setbeamerfont{block title}{series=\bfseries}
\setbeamercolor{block title}{fg=white, bg=Backdrop!105}
\setbeamercolor{block body}{fg=Foreground, bg=Backdrop!80}
\setbeamertemplate{items}[circle]
\setbeamercolor{item}{fg=Fabulous!50}
\setbeamerfont{footline}{size=\small}

\title{Modèle acteur et Scala}
\author{Merwan Achibet}
\institute{Université du Havre}
\date{Vendredi 24 février 2012}

\begin{document}

\maketitle

\section{Le modèle acteur}

\begin{frame}

  \begin{block}{Acteurs}
    Processus concurrent communiquant avec d'autres acteurs par
    échange de messages. Un acteur peut répondre à un message
    asynchrone en créant un nouvel acteur, en envoyant des messages ou
    en changeant de comportement. \cite{haller}
  \end{block}

  \vfill

  Idées conductrices :

  \begin{itemize}
  \item{Tout est acteur}
  \item{Asynchronisme}
  \item{Fault tolerance}
  \end{itemize}

\end{frame}

\begin{frame}

  \begin{block}{Motivations}
    \begin{itemize}
    \item{Parallélisation croissante du matériel}
    \item{Distribution des calculs}
    \item{Un paradigme structuré autour de ces idées}
    \end{itemize}
  \end{block}

  \vfill

  \begin{figure}

    \centering

    \begin{tikzpicture}

      \def \n {5}
      \def \radius {2cm}
      \def \margin {8}

      \foreach \s in {1,...,\n}
      {
        \node[draw, circle] (\s) at ({360/\n * (\s - 1)}:\radius) {$\s$};
        \draw[->, >=latex] ({360/\n * (\s - 1)+\margin}:\radius)
        arc ({360/\n * (\s - 1)+\margin}:{360/\n * (\s)-\margin}:\radius);
      }

      \draw (3) edge[->,>=latex,bend right] (1);
      \draw (1) edge[->,>=latex,bend right] (5);

    \end{tikzpicture}

    \end{figure}

\end{frame}

\begin{frame}

  \frametitle{Modèle acteur au sens strict \cite{rajesh}}

  \begin{block}{State encapsulation}
    Aucun partage de donnée hormis les messages.
  \end{block}

  \begin{block}{Safe-messaging}
    Les messages contiennent des copies strictes.
  \end{block}

  \begin{block}{Mobility}
    Le code et l'état d'un agent peuvent se déplacer entre
    processeurs, n\oe uds d'un réseau...
  \end{block}

  \begin{block}{Location transparency}
    Quelle que soit sa position, un agent dispose de la même adresse.
  \end{block}

\end{frame}

\begin{frame}

  \frametitle{JVM et concurrence}

  \begin{columns}

    \begin{column}{0.5\textwidth}
      \begin{block}{Thread-based}
        \begin{itemize}
        \item[+]{Simple}
        \item[-]{Lourd}
        \item[-]{Deadlock}
        \end{itemize}
      \end{block}
    \end{column}

    \begin{column}{0.5\textwidth}
      \begin{block}{Event-based}
        \begin{itemize}
        \item[+]{Performant}
        \item[-]{Vite tortueux}
        \end{itemize}
      \end{block}
    \end{column}

  \end{columns}

  \vfill

   \begin{block}{Une couche supplémentaire}
     Le langage Scala, de plus haut niveau, cache ces considérations
     techniques et permet de traiter la concurrence via le modèle
     agent.
   \end{block}

\end{frame}

\section{Le langage Scala}

\begin{frame}[containsverbatim]

  \begin{figure}
    \centering
    \includegraphics[width=2.5cm]{scala.png}
  \end{figure}

  \vfill

  \begin{block}{Origines}
    \begin{itemize}
    \item{Créé en 2003 à l'EPFL}
    \item{Académique}
    \item{Pragmatique}
    \item{Versatile}
    \end{itemize}
  \end{block}

  \vfill

  \begin{lstlisting}[frame=single]
    object HelloWorld {
      def main(args: Array[String]) {
        println("Hello, world!")
      }
    }
  \end{lstlisting}

\end{frame}

\begin{frame}[containsverbatim]

  \frametitle{Un langage orienté objet...}

  \lstinputlisting[frame=single]{personne.scala}

\end{frame}

\begin{frame}[containsverbatim]

  \frametitle{... mais aussi fonctionnel}

  \lstinputlisting[frame=single]{sort.scala}

\end{frame}

\begin{frame}[containsverbatim]

  \frametitle{Le modèle acteur avec Scala}

  \begin{lstlisting}[frame=single]
    class MonActeur extends Actor {

      ...

      def act() {

        ...

      }
    }
  \end{lstlisting}

  \vfill

  Un acteur hérite de la classe \texttt{Actor} et doit définir une méthode \texttt{act()}.

\end{frame}

\begin{frame}[containsverbatim]

  \frametitle{Communication entre acteurs}

  \begin{lstlisting}[frame=single]
    class E(r : Actor) extends Actor {
      def act() {
        receveur ! "Salut"
    ...
  \end{lstlisting}

  \vfill

  \begin{lstlisting}[frame=single]
    class R() extends Actor {
      def act() {
        loop {
          react {
            case message : String =>
              Console.println(message)
              sender ! "Bonjour"
        }
    ...
  \end{lstlisting}

\end{frame}

\begin{frame}

  \frametitle{Exemple de la partie de dés}

  \begin{block}{Règles du jeu}
    \begin{itemize}
    \item{Deux acteurs joueurs}
    \item{Un acteur arbitre}
    \item{le joueur qui tire le plus grand $n$ gagne ($1 \geq n \geq 6$)}
    \end{itemize}
  \end{block}

  \vfill

  \begin{block}{Déroulement de la partie}
    \begin{enumerate}
    \item{L'arbitre informe les joueurs du début de la partie}
    \item{Les joueurs tirent un $n$ au hasard et l'envoient à l'arbitre}
    \item{L'arbitre détermine le gagnant}
    \end{enumerate}
  \end{block}

\end{frame}

\begin{frame}[containsverbatim]

  On commence par définir les types de messages possibles.

  \vfill

  \lstinputlisting[frame=single]{de_partie1.scala}

\end{frame}

\begin{frame}[containsverbatim]

  Le comportement des joueurs :

  \vfill

  \lstinputlisting[frame=single]{de_partie2.scala}

\end{frame}

\begin{frame}[containsverbatim]

  Le comportement de l'arbitre :

  \vfill

  \lstinputlisting[frame=single]{de_partie3.scala}

\end{frame}

\begin{frame}[containsverbatim]

  La fonction principale du programme :

  \vfill

  \lstinputlisting[frame=single]{de_partie4.scala}

\end{frame}

\bibliographystyle{alpha}
\bibliography{references}

\end{document}
