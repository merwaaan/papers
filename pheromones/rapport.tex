\documentclass[12pt]{article}

\usepackage[utf8]{inputenc}
\usepackage[francais]{babel}
\usepackage{amsmath}
\usepackage{url}
\usepackage{float}

\title{Recherche de chemin par phéromones}
\author{Merwan Achibet $\;$ \textendash $\;$  Université du Havre}
\date{}

\begin{document}

\maketitle

\section*{Introduction}

On se propose d'étudier l'implémentation NetLogo de José M Vidal
inspirée de \textit{Synthetic Pheromone Mechanisms for Coordination of
  Unmanned Vehicles} de Parunak, Brueckner et Sauter. Dans un premier
temps, on étudie le modèle original puis on analyse
l'implémentation proposée.

\section{Modèle}

\subsection{Analogie avec le vivant}

Les fourmis sont des insectes reconnus pour leur caractère
social. Seules, elles restent vulnérables à l'environnement immense
les entourant. Pourtant, la coopération entre individus de la même
colonie caractérisant cette espèce leur permet de compter
parmi les êtres vivants les plus présents sur le globe terrestre.

Les fourmis communiquent par le biais de signaux chimiques que l'on peut assimiler à des odeurs, les phéromones
\cite{insectes}. Elles peuvent dialoguer d'individu à individu en
utilisant leurs antennes mais, dans le contexte de la croissance de la
fourmilière, il est plus efficace de déposer dans leur environnement
des messages généraux adressés à tous. C'est notamment le cas
lorsqu'une nouvelle source de nourriture est trouvée puisque que la
fourmi à l'origine de l'heureuse découverte déposera derrière elle des
phéromones dites de piste afin que ses congénères puissent y être
guidés. Plusieurs autres types de phéromones existent : certaines préviennent d'un
danger tandis que d'autre délimitent un territoire ou bien attirent les individus de l'autre sexe
en période de reproduction.

Les phéromones restant des signaux chimiques éphémères, un phénomène
d'évaporation se produit naturellement et efface les pistes les moins
arpentées (source de nourriture épuisée, mauvaise piste). Au
contraire, les meilleures pistes sont renforcées par accumulation puisque
toutes les autres fourmis les arpentant les renforcent par le dépôt de
leur propres phéromones. Les signaux chimiques sont aussi soumis à une diffusion
dans l'environnement et les insectes disposent d'un odorat assez fin pour en retrouver 
la source.

Ce type de comportement est une source d'inspiration pour la
conception de certaines méthodes de résolution destinée à des systèmes
multi-agents puisqu'il rassemble des qualités notables
\cite{parunak} :

\begin{description}
\item[Diversité]{Une phéromone peut aussi bien indiquer une piste à suivre qu'un danger. Tout type de sémantique peut leur être associé}
  \item[Distribution]{Les signaux chimiques guidant les fourmis sont répartis
sur tout l'environnement}
  \item[Décentralisation]{Si une fourmi est mangée par un oiseau, cela
  n'impactera pas l'avenir de la fourmilière car l'individu manquant
  n'est qu'un rouage d'un mécanisme plus complexe}
  \item[Dynamicité]{Les tracés se renforçant et s'évaporant continuellement, les fourmis
s'adaptent à tout changement impromptu de l'environnement}
\end{description}

Parunak \textit{et al.} s'inspirent de ce constat, ainsi que des
modèles existants \cite{dorigo}, pour proposer une méthode de guidage
destinée aux véhicules non habités dans le cadre
d'opérations militaires.

\subsection{Les agents et leur monde}

Ici, la fourmi est remplacée par un véhicule, un drône aérien dans l'exemple. Sa mission
est de partir d'une base, et d'atteindre un objectif de type bâtiment ennemi tout en évitant
certaines infrastructures pouvant l'impacter négativement (radars ennemis, batteries anti-aérienne).

Dans la simulation, un véhicule peut déposer deux types de phéromones :

\begin{description}
  \item[GTarget]{Libérée par un agent venant de rencontrer une cible ennemi et rentrant à sa base, elle guide les autres agents vers la cible.}
  \item[GNest]{Libérée par un agent venant de quitter la base, elle guide les agents vers la base.}
\end{description}

Ces deux signaux disposeront d'un taux de diffusion faible car les routes tracées n'ont aucunement besoin d'être larges. BLABLA

Dans ce modèle, les batîments ennemis sont eux aussi agents et émetteurs des phéromones :

\begin{description}
  \item[RTarget]{Libérée par les batîments cibles, elles attirent les drones}
  \item[RThreat]{Libérée par les batîments de contre-mesure, elles repoussent les drones}
\end{description}

Le taux de diffusion d'une phéromone dépend de l'aire d'effet du batîment associé, correspond à son rayon d'action; 
le rayon de détection pour un radar, la portée pour une batterie anti-aérienne.

L'environnement est divisé en zones (\textit{places}) dont les dimensions dépendent de la simulation envisagée. 
Dans l'exemple SEADy Storm proposé dans le papier étudié, une zone correspond à une hexagone de 50 kilomètres de diamètre.
Le travail d'une zone est de stocker une valeur réelle pour chaque type de phéromone y étant déposée mais aussi de les diffuser aux zones
voisines.

Le fonctionnement intrinsèque de ce modèle commence à s'esquisser. Les phéromones émisent par les
différents acteurs de la simulation vont tisser un canevas de signaux chimiques marquant l'environnement.
Typiquement, les drones auront pour but d'atteindre un bâtiment cible tout en évitant l'aire d'effet des
batîments menaçants.

Les phéromones décrites sont bien sûr spécifiques au modèle présenté et dans un soucis
de généralisme, on veut associer à chaque zone une fonction calculant sa valeur d'attractivité
en fonction des différentes phéromones qui y sont déposées. Ici, cette fonction est :

$$
\frac{\omega \operatorname{RTarget} + \gamma \operatorname{GTarget}}{\alpha \operatorname{RThreat} + \delta \operatorname{Dist}}
$$


\section{Implémentation}

\subsection{\'Etude}

\subsection{Analyse}

\bibliographystyle{alpha}
\bibliography{sources}

\end{document}
