\documentclass[12pt]{article}

\usepackage[utf8]{inputenc}
\usepackage[francais]{babel}
\usepackage{amsmath}
\usepackage{url}
\usepackage{float}

\title{Recherche de chemin par phéromones}
\author{Merwan Achibet $\;$ \textendash $\;$  Université du Havre}
\date{}

\begin{document}

\maketitle

\section*{Introduction}

On se propose d'étudier l'implémentation NetLogo de José M Vidal
inspirée de \textit{Synthetic Pheromone Mechanisms for Coordination of
  Unmanned Vehicles} de Parunak, Brueckner et Sauter. Dans un premier
temps, on étudiera le modèle original puis on analysera
l'implémentation proposée.

\section{Concept}

\subsection{Analogie avec le vivant}

Les fourmis sont des insectes reconnus pour leur caractère
social. Seules, elles restent vulnérables à l'environnement immense
les entourant. Pourtant, la coopération entre individus de la même
fourmilière qui caractérise cette espèce leur a permis de compter
parmi les êtres vivants les plus présents sur le globe terrestre.

Les fourmis communiquent par le biais de signaux chimiques appelés
phéromones et que l'on peut assimiler à des odeurs
\cite{insectes}. Elles peuvent dialoguer d'individu à individu en
utilisant leurs antennes mais, dans le contexte de la croissance de la
fourmilière, il est plus efficace de déposer dans leur environnement
des messages généraux adressés à tous. C'est notamment le cas
lorsqu'un nouvelle source de nourriture est trouvée puisque que la
fourmi à l'origine de l'heureuse découverte déposera derrière elle des
phéromones dites de piste afin que ses congénères puissent y être
guidés.

Les phéromones restant des signaux chimiques éphémères, un phénomène
d'évaporation se produit naturellement et efface les pistes les moins
arpentées (source de nourriture épuisée, mauvaise piste) alors qu'au
contraire les meilleures pistes sont renforcées par le dépôt de
phéromones de toutes les autres fourmis les arpentant.

Ce type de comportement est une source d'inspiration pour la
conception de certaines méthodes de résolution destinée à des systèmes
multi-agents puisqu'il rassemble des qualités notables
\cite{parunak} :

\begin{description}
  \item[Distribution]{}
  \item[Décentralisation]{Si une fourmi est mangée par un oiseau, cela
  n'impactera pas l'avenir de la fourmilière car l'individu manquant
  n'est qu'un rouage d'un mécanisme complexe}
  \item[Dynamicité]{}
\end{description}

Parunak \textit{et al.} s'inspirent de ce constat, ainsi que des
modèles existants (CITATIONS), pour proposer une méthode de guidage
destinée aux véhicules non habités, plus spécifiquement dans le cadre
d'opérations militaires.



\subsection{Champ de potentiel}

\subsection{Les agents et leur monde}

\section{Implémentation}

\subsection{\'Etude}

\subsection{Analyse}

\bibliographystyle{alpha}
\bibliography{sources}

\end{document}
