\documentclass[10pt]{article}

\usepackage[utf8]{inputenc}
\usepackage[francais]{babel}
\usepackage{multicol}
\usepackage{float}
\usepackage{tikz}

% FAMFAMFAM.com colors
\definecolor{mpink}{RGB}{255,11,91}
\definecolor{mblue}{RGB}{11,206,255}
\definecolor{mgreen}{RGB}{123,255,45}

\setlength{\textwidth}{39pc}
\setlength{\textheight}{54pc}
\setlength{\parindent}{1em}
\setlength{\parskip}{0pt plus 1pt}
\setlength{\oddsidemargin}{0pc}
\setlength{\marginparwidth}{0pc}
\setlength{\topmargin}{-8pc}
\setlength{\headsep}{20pt}
\setlength{\columnsep}{2pc}

\title{Recherche décentralisée de connexité pour réseaux de capteurs mobiles}
\author{Merwan Achibet}
\date{}

\begin{document}

\maketitle

\begin{multicols}{2}

\section{Introduction}

On imagine le problème suivant : un groupe de $n$ capteurs mobiles est
réparti aléatoirement dans un espace à deux dimensions. Un capteur ne
peut communiquer avec un autre capteur que si la distance les séparant
est inférieure à un seuil donné.

Cette contrainte de communication nous oblige à former un réseau
connexe de capteurs car on souhaite que toute information puisse être
partagée de capteur en capteur. Spatialement, cela signifie que l'on
doit positionner les capteurs afin qu'ils appartiennent tous à la même
composante connexe via leur rayon de communication. Cependant, une
agglomération naïve de tous les capteurs dans une zone est inadapté à
ce problème car on souhaite que les capteurs couvrent une zone
importante, leur but étant avant tout de relever le plus de données
possibles.

La décentralité du réseau implique que chaque capteur n'a pas de vue
globale du système mais peut construire progressivement une vue, au
début partielle et approximative puis plus précises, à l'aide des
données de ses voisins.

Deux questions interdépendantes se posent donc :

\begin{enumerate}
  \item{Comment déterminer de manière décentralisée la connexité du
    réseau ?}
  \item{Comment déplacer les capteurs pour rendre le réseau connexe ?}
\end{enumerate}

Dans la section 2, nous nous attelerons à proposer un algorithme
simple permettant d'évaluer si un réseau de capteurs est connexe de
manière décentralisée. Ensuite, nous envisagerons une méthode de
guidage des capteurs inspirées des boïds et des systèmes
particulaires.

\begin{figure}[H]

  \centering

  \begin{tikzpicture}

  \draw (0,0) -- (-1,1);

  \drawSensor{0}{0}{2}{red}
  \drawSensor{-1}{1}{2}{blue}
  \drawSensor{2.5}{1}{2}{green}

\end{tikzpicture}


  \caption{}
  \label{}

\end{figure}

\section{Déterminer si le réseau est connexe}

\section{Rendre le réseau connexe}

On présente une méthode de guidage décentralisée permettant au
déplacement de chaque capteur de participer à l'émergence d'une
connexité globale.

Ce guidage, inspiré des boïds \cite{Reynolds1987} et des essaims
particulaires gazeux \cite{Cheng2011497}, est dépendant de trois
forces combinées en un seul mouvement. La première force,
l'attraction, va associer à chaque capteur une force attirant les
autres capteurs. La seconde force, la répulsion, va assurer que les
capteurs ne soit pas tous regroupés dans la même zone et permet donc
d'occuper un espace plus large. La dernière force, la gravité, permet
de donner, aussi bien aux capteurs isolés qu'aux composantes connexes
de capteurs, un mouvement global vers le center de l'espace à
analyser.

\subsection{Position connue ou approximée}

Puisque l'on souhaite que chaque capteur se déplace indépendamment
tout en participant à une évolution commune du réseau menant à la
connexité, la connaissance de sa position propre ainsi que de la
position de ses voisins est importante. Deux hypothèses sont
envisageables : 1) les capteurs sont équippés pour connaître leur
position absolue dans l'espace 2) les capteurs n'ont pas connaissance
de leur position absolue.

Dans le premier cas, la tâche est plus aisée car un capteur peut de
lui même se rendre à une position donnée, grâce à une position propre
connue et précise. Cette hypothèse est envisageable puisqu'avec la
diminution continue de la taille des composants électroniques, des
instruments de localisation pourraient facilement être embarqués dans
un capteur.

Dans le second cas, TRILATERATION !

\subsection{Attraction}

\`A la lecture de ce problème, la nécessité de rapprocher les capteurs
les uns des autres afin qu'un réseau de communication ininterrompu se
forme vient naturellement à l'esprit. En effet, la condition
\textit{sine qua non} au bon fonctionnement du réseau est la
communication, un capteur isolé est inutile, puisque son information
n'est pas partagée.

Pour les calculs de positionnement qui suivront, on part de
l'hypothèse qu'un capteur $C$ peut obtenir les positions exactes de
ses voisins et qu'il connaît la sienne.

On associe à $C$ un rayon d'attraction $R_a$ donc le but sous-jacent
est d'attirer les capteurs se trouvant à la fois dans le rayon de
communication de $C$ et à l'extérieur de $R_a$.

\begin{figure}[H]

  \centering

  \begin{tikzpicture}

  \drawSensor{0}{0}{2}{black}
  \draw (0,0) circle (1.5);

  \draw[fill=black] (200:1.8) circle (0.1);
  \draw[->] (200:1.8) -- (200:1);

\end{tikzpicture}


  \caption{}
  \label{}

\end{figure}

$$
\vec{a} = \vec{p}_c - \vec{p}_v
$$

\subsection{Répulsion}

\begin{figure}[H]

  \centering

  \begin{tikzpicture}

  \drawSensor{0}{0}{2}{black}
  \draw (0,0) circle (1.5);

  \draw[fill=black] (150:0.5) circle (0.1);
  \draw[->] (150:0.5) -- (150:1.3);

\end{tikzpicture}


  \caption{}
  \label{}

\end{figure}

$$
\vec{r} =
$$

\subsection{Gravité}

\begin{figure}[H]

  \centering

  \begin{tikzpicture}

  \node[path picture={
      \draw[black] (path picture bounding box.south east) -- (path
      picture bounding box.north west);
      \draw[black] (path picture bounding box.north east) -- (path
      picture bounding box.south west);}] at (0,0) {};

  \draw[fill=black] (120:3) circle (0.1);
  \draw[->] (120:3) -- (120:2.5);

  \draw[fill=black] (180:3) circle (0.1);
  \draw[->] (180:3) -- (180:2.5);

  \draw[fill=black] (10:3) circle (0.1);
  \draw[->] (10:3) -- (10:2.5);

\end{tikzpicture}


  \caption{}
  \label{}

\end{figure}

$$
\vec{g} =
$$

\subsection{Composition d'une force nette}


\end{multicols}

\bibliographystyle{alpha}
\bibliography{references}

\end{document}
