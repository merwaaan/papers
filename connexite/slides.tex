\documentclass{beamer}

\usepackage[utf8]{inputenc}
\usepackage[francais]{babel}
\usepackage{tikz}

% FAMFAMFAM.com colors
\definecolor{mpink}{RGB}{255,11,91}
\definecolor{mblue}{RGB}{11,206,255}
\definecolor{mgreen}{RGB}{123,255,45}

\usetheme{Warsaw}

\setbeamertemplate{navigation symbols}{}
\setbeamertemplate{footline}[frame number]

\definecolor{Backdrop}{RGB}{85, 98, 112}
\definecolor{Foreground}{RGB}{238, 238, 238}
\definecolor{Fabulous}{RGB}{210, 111, 127}

\usecolortheme[named=Backdrop]{structure}
\setbeamerfont{title}{series=\bfseries}
\setbeamercolor{title}{bg=Backdrop}
\setbeamercolor{frametitle}{fg=white, bg=Backdrop}
\setbeamercolor{normal text}{fg=Backdrop, bg=white}
\setbeamerfont{alerted text}{series=\bfseries}
\setbeamercolor{alerted text}{fg=Backdrop!120}
\setbeamertemplate{blocks}[rounded][shadow=true]
\setbeamerfont{block title}{series=\bfseries}
\setbeamercolor{block title}{fg=white, bg=Backdrop!105}
\setbeamercolor{block body}{fg=Foreground, bg=Backdrop!80}
\setbeamertemplate{items}[circle]
\setbeamercolor{item}{fg=Fabulous!50}
\setbeamerfont{footline}{size=\small}

\title{Recherche décentralisée de connexité pour réseaux de capteurs mobiles}
\author{Merwan Achibet}
\institute{Université du Havre}
\date{Jeudi 16 février 2012}

\begin{document}

\maketitle

\section{Introduction}

\begin{frame}

  \begin{block}{Problème}
    \begin{itemize}
    \item{Des capteurs mobiles sont lâchés dans un espace}
    \item{Pour communiquer, il doivent être proches}
    \item{Pour être efficaces, ils doivent être dispersés}
    \end{itemize}
  \end{block}

  \vfill

  Deux questions :
  \begin{enumerate}
  \item{Comment savoir si le réseau est connexe ?}
  \item{Comment rendre le réseau connexe ?}
  \end{enumerate}

  \vfill

  \alert{Et ce, de manière décentralisée !}

\end{frame}

\newcommand{\tikzbullet}[1]{
  \begin{tikzpicture}
    \draw[fill=#1,#1] (0,0) circle (0.1);
  \end{tikzpicture}
}

\begin{frame}

  \begin{figure}
    \begin{tikzpicture}

  \draw (0,0) -- (-1,1);

  \drawSensor{0}{0}{2}{red}
  \drawSensor{-1}{1}{2}{blue}
  \drawSensor{2.5}{1}{2}{green}

\end{tikzpicture}

  \end{figure}

  \vfill

  \begin{itemize}
  \item[$\blacktriangleright$]{\tikzbullet{mpink} et \tikzbullet{mblue} sont connectés}
  \item[$\blacktriangleright$]{\tikzbullet{mgreen} est isolé}
  \end{itemize}

\end{frame}

\section{Déterminer la connexité}

\begin{frame}

\end{frame}

\section{Créer la connexité}

\begin{frame}

  \frametitle{Inspirations}

  \begin{block}{Boids}
    \begin{itemize}
    \item{Craig W. Reynolds, 1987}
    \item{Un jeu de règles simple}
    \item{Les actions locales...}
    \item{... aboutissent à un comportement global}
    \end{itemize}
  \end{block}

  \vfill

  \begin{block}{Systèmes particulaires}
    \begin{itemize}
    \item{Cheng, Cheng et Nagpal, 2005}
    \item{Forces de répulsion}
    \item{Répartition de particules dans des formes géométriques}
    \end{itemize}
  \end{block}

\end{frame}

\begin{frame}

  \frametitle{L'attraction}

  \begin{figure}
    \begin{tikzpicture}

  \drawSensor{0}{0}{2}{black}
  \draw (0,0) circle (1.5);

  \draw[fill=black] (200:1.8) circle (0.1);
  \draw[->] (200:1.8) -- (200:1);

\end{tikzpicture}

  \end{figure}

\end{frame}

\begin{frame}

  \frametitle{La répulsion}

  \begin{figure}
    \begin{tikzpicture}

  \drawSensor{0}{0}{2}{black}
  \draw (0,0) circle (1.5);

  \draw[fill=black] (150:0.5) circle (0.1);
  \draw[->] (150:0.5) -- (150:1.3);

\end{tikzpicture}

  \end{figure}

\end{frame}

\begin{frame}

  \frametitle{La gravité}

  \begin{figure}
    \begin{tikzpicture}

  \node[path picture={
      \draw[black] (path picture bounding box.south east) -- (path
      picture bounding box.north west);
      \draw[black] (path picture bounding box.north east) -- (path
      picture bounding box.south west);}] at (0,0) {};

  \draw[fill=black] (120:3) circle (0.1);
  \draw[->] (120:3) -- (120:2.5);

  \draw[fill=black] (180:3) circle (0.1);
  \draw[->] (180:3) -- (180:2.5);

  \draw[fill=black] (10:3) circle (0.1);
  \draw[->] (10:3) -- (10:2.5);

\end{tikzpicture}

  \end{figure}

\end{frame}

\begin{frame}

  \frametitle{Combiner les différentes influences}

  \begin{equation}
    \vec{f} = \frac{\vec{a} + \vec{r} + \vec{g}}{3}
  \end{equation}

  \vfill

  \begin{block}{}
    \begin{itemize}
    \item{Chaque force a la même importance}
    \item{Au début de la simulation, acceptable}
    \item{Ensuite, le maillage s'affaisse}
    \end{itemize}
  \end{block}

  \vfill

  $\rightarrow$ Démonstration

\end{frame}

\begin{frame}

  \frametitle{Combiner les différentes influences}

  \begin{block}{Prioritiser les forces}

    \begin{itemize}
    \item{Répulsion $\rightarrow$ Attration $\rightarrow$ Gravité}
    \item{Chaque capteur a une vitesse maximale}
    \end{itemize}

    \begin{enumerate}
      \item{On applique la 1ère force}
      \item{S'il reste de la magnitude, on applique la 2nde force}
      \item{S'il reste de la magnitude, on applique la 3ème force}
    \end{enumerate}
  \end{block}

  \vfill

  $\rightarrow$ Démonstration

\end{frame}

\begin{frame}

  \frametitle{Test avec répartition sérrée}

\end{frame}

\begin{frame}

  \frametitle{Test avec répartition large}

\end{frame}

\end{document}
